\documentclass[12pt]{article}

\usepackage{pictex}
\usepackage{multirow}
\usepackage{color}
\usepackage[pdftex]{graphicx}
\usepackage{fancyhdr}
\usepackage[portuguese]{babel}
\usepackage[utf8]{inputenc}
\usepackage{xcolor,colortbl}
\usepackage{placeins}
\usepackage{verbatim}
\usepackage{geometry}
\usepackage{lscape}
\usepackage{listings}
\usepackage{color}
\usepackage{times}
\usepackage{afterpage}
\usepackage{hyperref}



\definecolor{dkgreen}{rgb}{0,0.6,0}
\definecolor{gray}{rgb}{0.5,0.5,0.5}
\definecolor{mauve}{rgb}{0.58,0,0.82}

\lstset{frame=tb,
  language=Java,
  aboveskip=3mm,
  belowskip=3mm,
  showstringspaces=false,
  columns=flexible,
  basicstyle={\small\ttfamily},
  numbers=none,
  numberstyle=\tiny\color{gray},
  keywordstyle=\color{blue},
  commentstyle=\color{dkgreen},
  stringstyle=\color{mauve},
  breaklines=true,
  breakatwhitespace=true,
  tabsize=3
}

\newcommand{\HRule}{\rule{\linewidth}{0.2mm}}


 \geometry{
 total={210mm,297mm},
 left=30mm,
 right=30mm,
 top=30mm,
 bottom=30mm,
 }
 
 \pagestyle{fancy}
\fancyhf{}
\lhead{\footnotesize Software Requirements Specification para Web dashboard para o git}
\rhead{\footnotesize 30 de Setembro de 2016}
\lfoot{}
\cfoot{\footnotesize Página \thepage}
\rfoot{}
 
 
 
\begin{document} 
\begin{titlepage}
\begin{center}

\textsc{\footnotesize Universidade de Coimbra }\\[0.1cm]
\textsc{\footnotesize Faculdade de Ciências e Tecnologias}  \\[0.1cm]
\textsc{\footnotesize Departamento de Engenharia Informática }\\[0.2cm]
\textsc{\footnotesize Engenharia de Software }\\[2cm]




{\Large \bfseries Software Requirements Specification\\[1.5cm]

\textsc{\normalsize Version 1}\\[.2cm]
\textsc{\normalsize Web Dashboard para o Git}\\[2cm]
\textsc{\fontsize{2cm}{2.5cm}\selectfont EdgeSoft}\\[5cm]


\small
André Duarte \hfill 2014214845\\ 
Diana Pereira \hfill 2013150712\\
Gonçalo Pinto \hfill 2014202176\\
Henrique Pereira \hfill 2011154576\\
João Costa \hfill 2013144034\\
João Ferreiro \hfill 2014197760\\
Joel Pires \hfill 2014195242\\



\begin{minipage}{0\textwidth}
\begin{flushright} \large
\end{flushright}
\end{minipage}

\vfill
{\normalsize 30 de Setembro de 2016}
}

\end{center}
\end{titlepage}


\pagebreak

\tableofcontents

\pagebreak

\section{ \textsc{Introdução}}

\subsection{ \textsc{Objectivo}}
Este documento especifica os requisitos de software para todo o sistema de uma web dashboard do Git que funcione como mentor e guia virtual de projetos guardados no repositório do DEI (https://git.dei.uc.pt.).

\subsection{ \textsc{Convenções do Documento}}
\textbf{Titulo}
Font: Helvetica Neue
Face: Bold
Size: 25

\textbf{Sub-títulos}
Font: Helvetica Neue
Face: Bold
Size: 12

\textbf{Outro Texto}
Font: Helvetica Neue
Size: 11

\subsection{ \textsc{Público-alvo e sugestões de leitura}}
Pretende-se que o presente documento seja lido pelos desenvolvedores do software, por testers e pelo cliente.

\subsection{ \textsc{Âmbito do Projecto}}
Este projeto consiste numa web dashboard do Git capaz de listar os vários documentos, elaborar estatísticas, executar tarefas, processos e outros artefactos relativos a um projecto alojado na plataforma do Gitlab, de forma a servir de mentor e ajuda a quem está a desenvolver. Esse mentor, consoante os requisitos e processos que a equipa quer usar, é capaz de sugerir tarefas a realizar, estimar tempos, boas práticas de programação, aconselhar determinada decisão a tomar, avaliar o desempenho de cada membro e a qualidade do projecto. No entanto, tal mentor não deve ser de uso obrigatório. O objetivo é ajudar uma equipa a estar consciente de todos os aspetos do seu projeto de modo a garantir um bom progresso.

\subsection{ \textsc{Referências}}
\begin{itemize}
\item Documento de especificação de requisitos de software
\item Material fornecido pelo docente da cadeira: IEEE Software Requirements Specification Template 
\end{itemize}

\pagebreak

\section{ \textsc{Descrição Geral}}
\subsection{ \textsc{Perspectiva do Produto}}
O projecto surge no âmbito da disciplina de Engenharia de Software, que faz parte dos cursos de Engenharia Informática e Design e Multimédia. Ao tratar-se de uma aplicação capaz de analisar as várias componente de projectos num repositório Git do DEI, servirá para melhorar os seus desenvolvimentos pelas equipas que usem este produto

\subsection{ \textsc{Funções do Produto}}
Providencia Informações Relativas ao Projeto:
\begin{itemize}
\item Hierarquia de Ficheiros
\item Membros da Equipa
\item Contribuintes para o Projeto
\item Linguagens de Programação Presentes
\item Tamanho do Projeto
\item Informação sobre Forks
\item Informação sobre Pull Requests
\item Informação sobre Commits
\item Informação sobre Inserções/Deleções
\item Informação sobre Issues
\item Informação sobre Branches
\item Informação sobre Releases
\item Informação sobre Followers
\item Informação sobre Stargazers
\item Informação sobre Downloads
\item Informação sobre Comentários
\end{itemize}

\pagebreak

Providência Estatísticas Relativas ao Projeto:
\begin{itemize}
\item Visão temporal das atividades (quantidade de commits, pull requests,etc) de cada membro
\item Comparação dessas atividades entre os membros da equipa
\item Visão temporal das atividades (quantidade total de commits, pull requests, etc) do projeto
\item Comparação entre várias referências temporais das atividades do projeto
\item Análise do tempo gasto para metas e para o projeto na sua íntegra,  quer por ada membro da equipa quer a totalidade da equipa
\end{itemize}

\hfill \linebreak

rovidência supervisionamento do projeto (mentor do projeto):
\begin{itemize}
\item sugere tarefas a realizar
\item prioriza e agenda tarefas a realizar
\item aconselha boas práticas de programação
\item aconselha sobre decisões que são precisas tomar
\item lança alertas dependendo da urgência das metas
\item faz controle do tempo gasto e disponível para as metas a atingir
\end{itemize}

Dispõe de Opções de configuração e Preferências do Utilizador
\begin{itemize}
\item autêntica membro do projeto por meio de um sistema de log in
\item mentor do projeto é configurável 
\item é possível adicionar processo usado pela equipa
\item modificar perfil dos membros
\end{itemize}

\subsection{ \textsc{Classes de Utilizador e Características}}
A aplicação destina-se apenas utilizadores do repositório Git do DEI, os quais devem fazer parte de uma equipa a desenvolver um projecto nesse mesmo repositório

\subsection{ \textsc{Ambiente Operacional}}
Este projeto vai ser operacional em qualquer sistema operativo e browser

\pagebreak

\section{ \textsc{Funcionalidades do Sistema}}
\subsection{ \textsc{Navegar pela hierarquia do projeto}}

\begin{center}
\begin{tabular}{ | m{4cm} | m{10cm} | } 
\hline
ID Caso de Uso & 1.1 \\
\hline
Nome & Navegar pela hierarquia do projecto\\
\hline
Descrição & O utilizador pode ver a listagem ou aceder todos os ficheiros e pastas do projeto do repositório git\\
\hline
Pré-Requesitos & O utilizador deve estar autenticado\\
\hline
Pós-Requesitos & É mostrada uma interface com a listagem do conteúdo do projecto (directoria ou conteúdo do ficheiro)\\
\hline
Caso de Falha & O utilizador não consegue visualizar o conteúdo do projecto\\
\hline
Actor & Operador\\
\hline
Desencadeado & O utilizador tem que escolher a opção Ficheiros\\
\hline
\end{tabular}
\end{center}


\subsection{ \textsc{Informações gerais do Projeto}}

\begin{center}
\begin{tabular}{ | m{4cm} | m{10cm} | } 
\hline
ID Caso de Uso & 1.2 \\
\hline
Nome & Informações gerais do Projeto\\
\hline
Descrição & É possível ao utilizador ter acesso a todas as informações relativamente ao projecto (commits, branches, pull requests, Inserções/Deleções, Releases, ...)\\
\hline
Pré-Requesitos & O utilizador deve-se encontrar autenticado\\
\hline
Pós-Requesitos & Uma listagem da informação em relação aos vários tópicos referidos em cima\\
\hline
Caso de Falha & O utilizador não consegue visualizar as informações gerais do projecto\\
\hline
Actor & Operador\\
\hline
Desencadeado & O utilizador escolhe a opção Projecto\\
\hline
\end{tabular}
\end{center}


\subsection{ \textsc{Informações gerais sobre Membros da Equipa}}

\begin{center}
\begin{tabular}{ | m{4cm} | m{10cm} | } 
\hline
ID Caso de Uso & 1.3\\
\hline
Nome & Informações gerais sobre Membros da Equipa\\
\hline
Descrição & Obter os dados relativamente ao nome, foto, idade
 e curso de cada membro \\
\hline
Pré-Requesitos & Estar autenticado como membro da equipa\\
\hline
Pós-Requesitos & É mostrada uma interface com a informação de
 cada um dos membros da equipa\\
\hline
Caso de Falha & O utilizador não consegue ver alguns ou
 todos os membros da equipa\\
\hline
Actor & Operador\\
\hline
Desencadeado & O utilizador escolhe a opção Membros \\
\hline
\end{tabular}
\end{center}


\subsection{ \textsc{Estatísticas sobre Atividade no Projeto}}

\begin{center}
\begin{tabular}{ | m{4cm} | m{10cm} | } 
\hline
ID Caso de Uso & 1.4\\
\hline
Nome & Estatísticas sobre Atividade no Projeto\\
\hline
Descrição & 
É possível ao utilizador consultar as estatísticas do projecto, ao longo de todo o curso do mesmo, em relação às informações gerais do projecto entre os vários membros
\\
\hline
Pré-Requesitos & O utilizador deve estar autenticado\\
\hline
Pós-Requesitos & É apresentada a informação em relação às várias estatísticas do projecto \\
\hline
Caso de Falha & A informação em relação às estatísticas não é apresentada\\
\hline
Actor & Operador\\
\hline
Desencadeado & O utilizador escolhe a opção de estatísticas no menu\\
\hline
\end{tabular}
\end{center}


\subsection{ \textsc{Estatísticas sobre Tempo Gasto no Projeto}}

\begin{center}
\begin{tabular}{ | m{4cm} | m{10cm} | } 
\hline
ID Caso de Uso & 1.5\\
\hline
Nome & Estatísticas sobre Tempo Gasto no Projeto\\
\hline
Descrição & Obter os dados relativamente ao tempo gasto de todos os membros na realização do projecto \\
\hline
Pré-Requesitos & O utilizador deve estar autenticado como membro da equipa\\
\hline
Pós-Requesitos & É mostrada uma interface com a informação do tempo gasto no projeto pelos membros do grupo\\
\hline
Caso de Falha & A informação em relação ao tempo gasto no projecto não é apresentada\\
\hline
Actor & Operador\\
\hline
Desencadeado & O utilizador escolhe a opção de estatísticas no menu\\
\hline
\end{tabular}
\end{center}
------------------------------

\subsection{ \textsc{Controle do Tempo Gasto}}

\begin{center}
\begin{tabular}{ | m{4cm} | m{10cm} | } 
\hline
ID Caso de Uso & 1.6\\
\hline
Nome & Controle do Tempo Gasto\\
\hline
Descrição & Obter a informação relativamente ao budget temporal do utilizador\\
\hline
Pré-Requesitos & O utilizador tem de estar autenticado\\
\hline
Pós-Requesitos & É apresentada informação em relação ao budget (budget total, budget usado por semana, …)\\
\hline
Caso de Falha & A informação em relação ao budget temporal do projecto não é apresentada\\
\hline
Actor & Sistema\\
\hline
Desencadeado & O utilizador carrega no seu avatar no menu lateral esquerdo\\
\hline
\end{tabular}
\end{center}


\subsection{ \textsc{Autenticação de Utilizadores}}

\begin{center}
\begin{tabular}{ | m{4cm} | m{10cm} | } 
\hline
ID Caso de Uso & 1.7\\
\hline
Nome & Autenticação de Utilizadores\\
\hline
Descrição & O utilizador pode introduzir o seu Token privado da API do git do DEI para aceder às funcionalidades da dashboard\\
\hline
Pré-Requesitos & O utilizador deve estar autenticado\\
\hline
Pós-Requesitos & Mostrar o menu incial da dashboard\\
\hline
Caso de Falha & O utilzador não consegue usar as funcionalidade da dashboard\\
\hline
Actor & Operador\\
\hline
Desencadeado & O utilizador tem que escolher a opção Conectar\\
\hline
\end{tabular}
\end{center}


\subsection{ \textsc{Armazenamento das Preferências e Configurações do Utilizador}}

\begin{center}
\begin{tabular}{ | m{4cm} | m{10cm} | } 
\hline
ID Caso de Uso & 1.8\\
\hline
Nome & Armazenamento das Preferências e Configurações do Utilizador\\
\hline
Descrição & O utilizador pode alterar diversas opções pessoas da forma como a dashboard deve funcionar\\
\hline
Pré-Requesitos & O utilizador deve estar autenticado\\
\hline
Pós-Requesitos & Mostrar as opções alteradas e o seu sucesso\\
\hline
Caso de Falha & O utilizador não consegue mudar as sua opções preferenciais\\
\hline
Actor & Operador\\
\hline
Desencadeado & O utilizador tem que escolher a opção Preferências\\
\hline
\end{tabular}
\end{center}


\subsection{ \textsc{Priorização e agendamento de tarefas/metas a realizar}}

\begin{center}
\begin{tabular}{ | m{4cm} | m{10cm} | } 
\hline
ID Caso de Uso & 1.9\\
\hline
Nome & Priorização e agendamento de tarefas/metas a realizar\\
\hline
Descrição & A informação em relação ao agendamento semanal de tarefas e a prioridade das mesmas é apresentada ao utilizador\\
\hline
Pré-Requesitos & O utilizador deve estar autenticado\\
\hline
Pós-Requesitos & É apresentada a informação ao utilizador na página inicial do projecto\\
\hline
Caso de Falha & Não apresentação da informação relativa ao agendamento\\
\hline
Actor & Sistema\\
\hline
Desencadeado & É apresentado ao utilizador na página inicial do projecto\\
\hline
\end{tabular}
\end{center}


\subsection{ \textsc{Lançamento de alertas}}

\begin{center}
\begin{tabular}{ | m{4cm} | m{10cm} | } 
\hline
ID Caso de Uso & 1.10\\
\hline
Nome & Lançamento de alertas\\
\hline
Descrição & Alertas sobre boas práticas de programação, decisões que são precisas tomar, avisos quando a deadline para determinada meta está a chegar (entre outras coisas), surgem no ecrã de forma oportuna\\
\hline
Pré-Requesitos & O utilizador deve estar autenticado\\
\hline
Pós-Requesitos & Alertas são apresentadas em forma de pop-up no decorrer do uso do software\\
\hline
Caso de Falha & Alertas não são lançados\\
\hline
Actor & Sistema\\
\hline
Desencadeado & É apresentado ao utilizador no decorrer do uso do software\\
\hline
\end{tabular}
\end{center}


\subsection{ \textsc{Configuração do Mentor}}

\begin{center}
\begin{tabular}{ | m{4cm} | m{10cm} | } 
\hline
ID Caso de Uso & 1.11\\
\hline
Nome & Configuração do Mentor\\
\hline
Descrição & O utilizador pode fornecer os requisitos do projecto assim como o modelo de processo a usar usado para o Sistema saber como deve ajudar a equipa\\
\hline
Pré-Requesitos & O utilizador deve estar autenticado\\
\hline
Pós-Requesitos & Mostra se o Sistema interpretou com sucesso as opções\\
\hline
Caso de Falha & O sistema não consegue dar alertas e sugerir dicas para o projecto\\
\hline
Actor & Operador\\
\hline
Desencadeado & O utilizador tem que escolher a opção Configurar Mentor\\
\hline
\end{tabular}
\end{center}

\pagebreak

\end{document}



















